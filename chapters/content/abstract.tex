% ***************************************
% ***************************************
\chapter*{Abstract} \label{abstract}
\addcontentsline{toc}{chapter}{Abstract}
% ***************************************
% ***************************************

\begin{center}\large
    \textbf{\thesisTitle}\\
\end{center}

%%%%%%%%%%%%%
\begin{center}
\end{center}
%%%%%%%%%%%%%

\begin{center}
    \authorName\\
    \schoolName, \thesisYear\\
\end{center}

\begin{center}
    Co-advisor: \firstAdvisorName\\
    Co-advisor: \secondAdvisorName\\
\end{center}


%%%%%%%%%%%%%
\begin{center}
\end{center}
%%%%%%%%%%%%%






Alzheimer's disease (AD) is the leading form of dementia. This disease represents a heavy burden in families worldwide, as there is currently no cure, requires constant care and is highly correlated with increasing age. Over 25 million cases have been estimated worldwide and this number is predicted to increase two-fold every 20 years. Even though there are a variety of clinical markers available for the diagnosis of AD, the accurate and timely diagnosis of this disease remains elusive, and this detection can be too late to start planning or testing experimental treatments. Thus, a system that can reliably detect the formation of Alzheimer's years before symptoms begin to show could help identify individuals at risk that could benefit from these treatments. Recently, over a dozen of genetic variants predisposing to the disease have been identified by genome-wide association studies. However, these genetic variants only explain a small fraction of the estimated genetic component of the disease. Therefore, useful predictions of AD from genetic data could not rely on these markers exclusively as they are not sufficiently informative predictors. This Master in Computer Science thesis is aimed towards expanding the state of the art in the use of machine learning and deep learning algorithms to predict the genetic risk of Alzheimer's Disease by using a larger number of genetic variants. Experimental results indicate that the proposed models holds promise to produce useful predictions for clinical diagnosis of AD.

